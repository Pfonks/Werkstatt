
% ---
\documentclass[a4paper]{article}

% Packages
% ---
\usepackage{amsmath} % Advanced Math Typesetting
\usepackage[utf8]{inputenc} % Unicode support (Umlaute etc.)
\usepackage[ngerman]{babel} % Change hyphenation rules

\usepackage{amssymb}
\usepackage{hyperref} % Links
\usepackage{graphicx} % Vilder
\usepackage{listings} % Source code highlighting
\usepackage[inline]{enumitem}
\usepackage{fullpage} % weniger abstand zu den seiten
\usepackage{tabularx} % bessere Tabellen
\usepackage{pdfpages} % pdf einbinden mit \includepdf[pages={2}]{x.pdf}
\newcolumntype{L}[1]{>{\raggedright\arraybackslash}p{#1}} % linksbündig mit Breitenangabe
\newcolumntype{C}[1]{>{\centering\arraybackslash}p{#1}} % zentriert mit Breitenangabe
\newcolumntype{R}[1]{>{\raggedleft\arraybackslash}p{#1}} % rechtsbündig mit Breitenangabe
% mit C{2cm} eine Zentrierte 2cm berite spalte machen

\newlist{todolist}{itemize}{2}
\setlist[todolist]{label=$\square$}

 
%Code
\usepackage{color}
\usepackage{colortbl}
\usepackage{textcomp}
\definecolor{listinggray}{gray}{0.9}
\definecolor{lbcolor}{rgb}{0.95,0.95,0.95}
\lstset{
	backgroundcolor=\color{lbcolor},
	tabsize=4,
	rulecolor=,
	basicstyle=\fontsize{10}{10}, % Fontgröße erstes: Text, zweites: Zahlen
	upquote=true,
	aboveskip={1.5\baselineskip},
	columns=fixed,
	showstringspaces=false,
	extendedchars=true,
	breaklines=true,
	prebreak = \raisebox{0ex}[0ex][0ex]{\ensuremath{\hookleftarrow}},
	frame=single,
	showtabs=false,
	showspaces=false,
	showstringspaces=false,
	identifierstyle=\ttfamily,
	keywordstyle=\color[rgb]{0,0,1},
	commentstyle=\color[rgb]{0.133,0.545,0.133},
	stringstyle=\color[rgb]{0.627,0.126,0.941},
}
\begin{document}

\title{\vspace{-3cm}Zugangsbescheid\footnote{Dies ist eine temporäre Lösung bis das digitale Buchungssystem verfügbar ist.} \\
	Werkstatt C103}
\date{\vspace{-1cm}\today{}} % Sets date you can remove \today{} and type a date manually
\maketitle{} % Generates title
\section*{Einweisung}
Hiermit bestätige ich, \underline{\hspace{3cm}}, im Auftrag des Fachbereichs Informatik, dass \underline{\hspace{3cm}}, Matrikelnummer: \underline{\hspace{2cm}}, eine Einweisung in die Werkstatt erhalten hat.\\\\
\noindent
Diese Einweisung umfasst folgende Bereiche:
\begin{todolist}
	\item Allgemeine Benutzungs- und Sicherheitshinweise
	\item Lötstation
	\item Oszilloskop
	\item 3D-Drucker
	\item Standbohrmaschine 
	\item Proxxon
\end{todolist}

\section*{Schlüssel}
Ein Schlüssel für folgende Zeiträume übergeben: 
\begin{enumerate}
	\item \underline{\hspace{3cm}}
	\item \underline{\hspace{3cm}}
	\item \underline{\hspace{3cm}}
	\item \underline{\hspace{3cm}}
	\item \underline{\hspace{3cm}}
	\item \underline{\hspace{3cm}}
\end{enumerate}
\noindent
\section*{Benutzung und Haftung}
Geräte dürfen nur nach gültiger Einweisung benutzt / betrieben werden.\\
Das Betreten und die Benutzung der Werkstatt erfolgt auf eigene Gefahr. Alle Schäden werden vom Verursachenden getragen.\\
Im Falle eines Verlustes des Schlüssels ist eine Gebühr von 50€ fällig.
\vspace{1cm}\\
\noindent
\underline{\hspace{3cm}}\hspace{1cm} \underline{\hspace{3cm}}\\\\\\
\noindent
\underline{\hspace{3cm}}\hspace{1cm} \underline{\hspace{3cm}}\\\\\\

\end{document}
